
O estudo dos processos reprodutivos de uma sociedade são alvo de interesse de  diversos campos de conhecimento. Desde a formulação de políticas públicas para a área da saúde e mesmo previdenciárias, podem ser analisadas de múltiplas perspectivas, sociológicas, da biomedicina ou da demografia. A taxa de fecundidade total (TFT) e a taxa específica de fecundidade (TEF), por exemplo, são importantes indicadores para monitorar o crescimento da população e o padrão reprodutivo por idade, podendo direcionar políticas de acesso a direitos reprodutivos ou serem utilizadas para planejamento no dimensionamento de futuras políticas. A TFT e a TEF estão presentes na matriz de Indicadores e Dados Básicos (IDB) para a saúde no Brasil\footnote{Fonte: \href{http://tabnet.datasus.gov.br/tabdata/livroidb/2ed/indicadores.pdf}{Indicadores e Dados Básicos (IDB)}}.

No entanto, é naturalizado, em grande parte da demografia, que os aspectos reprodutivos de uma população devem ser analisados tendo as mulheres como principal fonte de informação \cite{watkins1993if}. Um pesquisador treinado por esse campo naturalmente irá interpretar a Taxa de Fecundidade Total, como a razão de filhos nascidos vivos, tidos por uma mulher ao final do seu período reprodutivo. 
Não obstante, não é novo o esforço para que os aspectos reprodutivos masculinos sejam também considerados pela demografia \cite{greene2000absent, goldscheider1996fertility}.

Alguns autores questionam que uma parte considerável dos processos reprodutivos são ignorados ao se negligenciar a fecundidade masculina (FM) e, ainda mais grave, tal esquecimento está associado muitas vezes ao fato de se considerar que as fecundidades masculina e feminina são iguais e se comportam da mesma forma e que, portanto, a primeira poderia ser ignorada sem maiores prejuízos. Todavia, estudos do século passado \cite{karmel1947relations, schoen1985population} já apontavam que tais diferenças existem e são importantes de serem consideradas. 	

Um exemplo ilustrativo da relevância do estudo da fecundidade masculina para a demografia é o problema dos dois sexos. Em 1932, Robert Kuczynski analisou as taxas de crescimento separadas para homens e mulheres na França do pós-Primeira Guerra, encontrando um valor positivo para os homens e um valor negativo para mulheres, o que indicaria, numa visão simplista, que a população francesa seria num futuro próximo, composta apenas por homens. Seu trabalho fundamenta um impasse clássico na demografia conhecido como problema dos dois sexos, que examina que tanto o sexo feminino, como o masculino tem um papel em determinar o crescimento total da população e deveriam ser ambos incluídos na abordagem clássica da teoria da população estável \cite{chung1994cycles}. Ainda hoje não há solução amplamente aceita para essa questão. 

Recentemente, grandes contribuições têm sido realizadas na área de estudo das taxas de fecundidade masculina \cite{zhang2010male, schoumaker2019male, joyner2012quality, daumler2016men}, porém, quando fazemos o recorte para a literatura nacional acerca da questão, apesar da produção científica de alta qualidade \cite{Wong1986, wong2022fecundidade,falcao_fecundidade_2013}, ainda há relativamente pouca literatura produzida a respeito. 

\citeonline{zhang2010male} e \citeonline{schoumaker2019male}, ao fazerem grandes compilados da FM em países distintos, evidenciam aspectos que diferenciam a fecundidade feminina da masculina. Entre outros, o padrão observado no ciclo reprodutivo, onde homens atingem o pico em idades mais avançadas e tem o período mais espalhado que as mulheres, sendo normalmente considerado o limite superior do período reprodutivo masculino até os 60 anos e o feminino até os 50. Adicionalmente, demonstram que em países com alta fecundidade, a diferença entre a fecundidade masculina e feminina tende a ser maior, enquanto, ao se aproximar da taxa de reposição, as TFTs dos dois sexos se aproximam, sendo observados países onde a taxa feminina fica levemente superior à masculina.

Porém, enquanto os estudos de \citeonline{schoumaker2019male} alcançam alguns países com sistemas de registros administrativos mais defasados (países emergentes) a partir da base do \textit{“demographic health survey”}, o trabalho de \citeonline{zhang2010male} é, em grande parte, baseados nos dados de registros administrativos de países desenvolvidos, onde se tem um histórico de consolidação dos sistemas. 

A qualidade dos registros é um dos grandes desafios que se impõe para os estudos da fecundidade masculina. Com relação aos homens como fonte de informações em inquéritos populacionais, \citeonline{zhang2010male} afirma que a subnotificação de filhos ocorre mais entre homens do que entre mulheres, principalmente entre jovens e fruto de relações extra-conjugais ou de casamentos anteriores. Porém, demonstra que em alguns casos, as informações masculinas podem ser mais precisas que aquelas dadas pelas suas parceiras e que utilizar mulheres como respondentes para informações sobre seus parceiros pode gerar erros. 

Ao utilizar registros administrativos, por outro lado, os pesquisadores se deparam com a questão das altas proporções de dados faltantes.  
\citeonline{dudel2021male}, analisando dados do período de 1970 a 2015 para países desenvolvidos, encontra proporções que variam entre menos de 1\% faltante (Suécia, 2002), até 47\% (Dinamarca, 1994) de omissões para idade paterna. As omissões podem ser devido a uma série de fatores, quando a mãe não conhece o pai ou não quer fornecer informações sobre ele, ou mesmo quando a informação não é considerada relevante pelos responsáveis pelo preenchimento do documento do nascido vivo. Os autores citam \cite{dudel2021male}, por exemplo, que na Alemanha, até 1999, a idade paterna era registrada apenas para nascidos vivos a partir de casais unidos formalmente.



No Brasil, os dados para calcular os indicadores relacionados à fecundidade, como a TFT e as TEFs femininas, são provenientes de censos, realizados pelo IBGE, assim como de registros administrativos consolidados no Sinasc - Sistema de Informações sobre Nascidos Vivos \cite{rede2002indicadores}.

Porém, é encontrada uma alta proporção de valores faltantes para informação da idade do pai nos registros administrativos do Sinasc, o que tem sido um grande empecilho para o estudo da temática no país. \citeonline{falcao_fecundidade_2013} se arriscou a utilizar a base para o estudo da temática, porém restringiu sua análise a 14 municípios onde a proporção de dado faltante para a idade do pai disponível no sistema era menor que 8\%. 



\begin{comment}
    \textcolor{blue}{colocar os problemas metodológicos, sub registro e métodos (imputação) e a falta de aplicação o Brasil. 
    Os resultados serão muito distintos a depender no método.
    Qual o nível e o padrão da fecundidade masculina no Brasil a partir de diferentes métodos de imputação? 
    
    perguntas secundária; 
     O que dizem os principais estudos sobre a fecundidade masculina no mundo e no Brasil?
     
     Quais os métodos que produzem os melhores resultados para a imputação da idade do pai?---Antes do objetivo, as perguntas de pesquisa.}
        
    
    \textit{\textbf{\textcolor{blue}{Buscar os estudos das preferências masculinas e sua importância para o futuro da fecundidade.}}}

\end{comment}





