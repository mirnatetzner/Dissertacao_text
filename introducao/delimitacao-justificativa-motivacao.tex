

A expectativa é que, a partir da imputação para a idade do pai, oriunda do Sistema de Informações sobre Nascidos Vivos, possam ser calculadas a taxa de fecundidade total e a taxa especifica de fecundidade masculina, visando traçar um histórico para a FM entre os anos 2012–2022 e propor uma metodologia que possibilite a investigação da temática da FM no Brasil. Complementarmente contribuir para ampliação e sistematização da documentação do DATASUS. Em consonância com a missão da Escola Nacional de Ciências Estatísticas (ENCE) de garantir o desenvolvimento de estudos sobre a dinâmica populacional e territorial e das condições de vida da população, englobando aspectos sociais, econômicos e ambientais no contexto brasileiro.

Destaca-se a relevância do estudo da fecundidade masculina para uma
compreensão global do fenômeno da transição demográfica, na exposição das
diferenças da transição de fecundidade entre os sexos associadas a composição da
pirâmide etária em determinado período, das diferentes durações dos períodos
reprodutivos entre homens e mulheres, assim como das diferenças de idade ao ter o primeiro filho. Trazer luz à forma como se dão esses processos no Brasil no período de 2012–2022 é de extremo valor no contexto de redução drástica da fecundidade feminina e envelhecimento populacional.

Espera-se que este trabalho contribua propondo modelos para imputar a idade do pai nos registros de declaração de nascido vivo para os anos propostos, por meio de métodos de imputação para dados faltantes. Assim, visamos contribuir para o debate dos aspectos reprodutivos que envolvem a fecundidade masculina no campo da demografia e das políticas públicas no Brasil.




\begin{comment}

    
    \textbf{\textit{\textcolor{blue}{Justificativa é com base na literatura,
    já a Motivação pode ter questões pessoais}}}
    ------------------
    
    
    \textcolor{blue}{Motivação}
    
    
    [Relevância para o cenário brasileiro]
    
    [envelhecimento da população] -- entendimento dos padrões de fecundidade masculina ajudam a entender as escolhas reprodutivas, força da escolha do pai por ter filhos?
    
    
    Portanto, justifica-se a relevância da temática no contexto da transição demográfica brasileira no sentido de que, para obtermos uma visão completa dos determinantes do fenômeno, é preciso analisar  a FM ao longo do tempo e nas diferentes regiões.

\end{comment}