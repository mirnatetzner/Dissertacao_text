
\textbf{Objetivo Geral:} analisar o padrão da fecundidade masculina no Brasil entre os anos de 2012 e 2022, avaliando distintos métodos de tratamento de informação faltante e discutindo acerca das potencialidades e limitações das diferentes abordagens.


\textbf{Objetivos Específicos:}

\begin{itemize}

         \item Levantar os principais estudos sobre fecundidade masculina no mundo e no Brasil, destacando as potencialidades e limitações das fontes de dados e métodos utilizados; 
        \item Discutir e aplicar diferentes métodos estatísticos para correção de problemas de dados faltantes de informações masculinas nos dados de nascimento;
        \item A analisar as tendências da fecundidade masculina no Brasil entre os anos de 2012 e 2022;
        \item Analisar o comportamento por idade da fecundidade masculina no período de 2012 a 2022, comparando as taxas de fecundidade total e por idade masculinas e femininas ao longo do período analisado.

\end{itemize}


Na primeira parte do trabalho será feita uma revisão dos principais tópicos acerca da fecundidade masculina no mundo e no Brasil. No segundo capítulo serão trazidas informações relevantes sobre a base de dados utilizada e da variável de interesse, a saber, o Sistema de Informações sobre Nascidos Vivos e a idade do pai. No terceiro capítulo detalharemos as abordagens metodológicas para o processo de imputação, assim como seus pressupostos e métricas de avaliação cabíveis. O quarto capítulo será composto de uma análise exploratória dos dados resultantes dos métodos de imputação. E, por último, no quinto e sexto capítulos serão feitas a análise dos resultados e a conclusão, apontando para a contribuição realizada por essa empreitada, elaborar uma metodologia capaz de lidar com o problema dos dados faltantes no registro da idade paterna no Sinasc. 