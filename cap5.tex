%%%%%%%%%%%%%%%%% DESENVOLVIMENTO %%%%%%%%%%%%%%%%%
\chapter{DESENVOLVIMENTO} % CAIXA ALTA

\section{Análise exploratória dos dados}

\textcolor{blue}{mOSTRAR OS DADOS DE IDADE DO PAI AO LONGO DO TEMPO POR REGIÃO}



\begin{comment}
    ■ R packages: mi, mice, and Amelia packages include features for
model checking with core functions for imputing missing values
(van Buuren & Groothuis-Oudshoorn, 2011; Honaker et al., 2011;

Su et al., 2011). VIM and miP packages visualise imputed data
(Brix, 2012; Templ et al., 2015). These packages function to
graphically compare distribution of observed and imputed data, by
offering scatterplots for plotting observed and imputed data against
another variable. The mi package has tools to produce residual
plots to check imputation models when imputing data using MICE.
Amelia software has a diagnostic feature called overimputation to
produce cross-validation plots of mean imputed values against
observed value with 90% confidence intervals (Honaker et al.,
2011).
\end{comment}


\begin{comment}
Guia --- Missing Data and Multiple Imputation Decision Tree (1):

1. Examine patterns of missingness - for entire scales, items within scales,
patterns of missing data points, groups of people
○ Univariate descriptive statistics
■ Check for any illogical values that could be missing data codes
■ Consider implications of different data codes (e.g., missing data for
“I prefer not to answer”, “not applicable”, or because someone
didn’t provide an answer each have different implications)

○ Multivariate descriptive statistics
■ Look for outliers and try to determine whether the data are “real”
responses or can be considered skip patterns (e.g., careless
responders or straight-line responders; King et al. 2018)
■ Look for missingness at many variables simultaneously to identify
missing data patterns. For example, people might be missing all of
the cognitive tasks but none of the survey data (some example
visualizations)

2. Evaluate how much missingness you have on each variable of interest
Remember to pay attention to the percent of both item-missing data (i.e.,
skipped questions on a survey) and attrition-missing data (i.e., people that
were missing data for the entire wave - sometimes indicated by skip codes
within datasets).

3. Choose a decision rule for determining whether missingness is related to other variables in your dataset

■ This decision rule would ideally be preregistered!
■ In large datasets, p values may not be a reliable method to determine
which variables meaningfully predict missingness patterns (i.e., there may
be statistically significant differences at p < .05 because analyses are
highly powered, but these differences may not be practically or
meaningfully important).
■ Effect sizes are a perfectly valid way to determine whether something is
related to the missingness.
● How big of an effect size is big enough to matter? It depends, of
course, on what the outcome is you’re interested in. (Again, this is
where preregistration comes in handy.) Pick an effect size that is
the smallest relevant effect (Sullivan & Feinn, 2012).
● If you are struggling with what a “smallest relevant effect” might be,
some suggest a general rule of r > 0.4 correlation as a threshold for
inclusion (e.g., Collins et al., 2001; Enders, 2010) as this value
seems to find a good balance between model parsimony and
increased statistical power. (Note: you may have a “smallest
relevant effect” that is r < 0.4 - that’s okay, but make sure you
document and justify why you expect this to be the case) (UCLA
Statistical Consulting Group).

4. Check whether any auxiliary variables are related to your missingness

Auxiliary variables are variables in your data set that are not of particular
interest to your analysis, but that are either correlated with a missing
variable(s) or are believed to be associated with missingness. Auxiliary

variables will be added to the imputation model to increase power and/or
to help make the assumption of MAR more plausible, because including
these variables has been found to improve the quality of imputed values
generated from multiple imputation. This is especially important when
imputing a dependent variable and/or when you have variables with a high
proportion of missing information (Johnson & Young, 2011; Young &
Johnson, 2010; Enders, 2010).
○ You should still check for potential auxiliary variables even if your data are
MCAR or MNAR!
■ With MCAR data, just double check to make sure that there is
nothing else predicting patterns of missingness.
■ With MNAR data, your sample will be biased no matter what.
However, you can lessen any additional bias and increase your
power by using MI and including auxiliary variables that may
explain other missingness patterns.

■ Test auxiliary variables for inclusion
1) Create a new dummy variable that represents whether your key
variable is missing (missdummy=1) or not missing (missdummy=0).
2) Then predict it with auxiliary variables you think are likely to be
related to missingness (e.g., t-tests for continuous auxiliary
variables, chi-square for binary auxiliary variables, anova for
categorical auxiliary variables). For example, in many longitudinal
academic datasets, variables like parent education, mobility,
income to needs ratio, and disability status are often related to
higher dropout (attrition) in longitudinal studies.
3) Repeat this for any key variables you plan to include in the analysis
(predictors; outcomes).

■ Note: Assume that any statistical MCAR tests are actually tests for
whether your data can be listwise deleted (van Ginkel et al., 2020); also,
see FAQ
\end{comment}




\begin{comment}
        \href{exemplo1}{https://hdsr.mitpress.mit.edu/pub/4tx7h11w/release/2}
    \href{exmplo2}{https://www.researchgate.net/publication/358529185_A_reinforcement_learning-based_approach_for_imputing_missing_data}
    \begin{algorithm}
    \caption{nome do método utilizado para a simulação }\label{alg:cap}
    \begin{algorithmic}
    \Require $n \geq 0$
    \Ensure $y = x^n$
    \State $y \gets 1$
    \State $X \gets x$
    \State $N \gets n$
    \While{$N \neq 0$}
    \If{$N$ is even}
        \State $X \gets X \times X$
        \State $N \gets \frac{N}{2}$  \Comment{This is a comment}
    \ElsIf{$N$ is odd}
        \State $y \gets y \times X$
        \State $N \gets N - 1$
    \EndIf
    \EndWhile
    \end{algorithmic}
    \end{algorithm}
\end{comment}

