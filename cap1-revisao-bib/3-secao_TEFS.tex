%Taxas Específicas de Fecundidade por idade

Quando falamos do padrão masculino da fecundidade nas faixas etárias, \apudonline{paget1994relational}{zhang2010male}, ao observar países com distintos níveis de TFT nas décadas de 1960-80, identificaram que os homens teriam, em geral, o início de sua vida reprodutiva postergado e uma parada bem posterior em relação às mulheres. Os homens teriam um pico mais tardio e inferior, com as TEFMs permanecendo mais elevadas do que as TEFFs em faixa-etárias mais avançadas. 

\citeonline{zhang2010male} subdivide a análise, com dados para os anos de 1990 a 1998, das TEFs entre países com TFT feminina alta e baixa (TFTM < 2,2 e TFTF < 2,2). O autor encontra que as TEFFs superam as TEFMs nas faixas etárias mais jovens (15–19, 20–24 e 25–29), com o valor médio da TEFFs da faixa etária de 15-19 anos sendo até cinco vezes maior que as TEFMs para esse grupo etário, tanto em países com alta, como com baixa TFT. A faixa de 30 a 34 anos se mostrou um ponto de inflexão na relação da fecundidade masculina com a feminina, sendo observado que nessa faixa as TEFMs começam a superar os níveis das TEFFs, com a razão da TEFM pela TFTF próxima de um. 

{\citeonline{zhang2010male} avança ao descobrir que um fator de mudança na correlação das TEFs feminina e masculina é quando as TFTs se aproximam do nível de reposição demográfica (2,2), quando a TFTM costuma alcançar um patamar inferior a TFTF. As hipóteses que o autor levanta sobre essa questão estão associadas ao denominador no cálculo das taxas de fecundidade, ou seja, mudanças no número de homens na faixa etária. O fato de países com TFT abaixo do nível de reposição serem, geralmente, países de economia desenvolvida pode estar associado a níveis altos de imigração, com o inverso verdadeiro, altas TFTs associadas a menor desenvolvimento e alta emigração. Compreendendo a migração como um movimento feito majoritariamente por homens jovens, isso contribuiria para que o denominador ficasse maior (e uma menor TFTM) em países desenvolvidos e menor em países menos desenvolvidos (TFTM maior). A segunda hipótese, também atrelada aos níveis de desenvolvimento, foca nos diferenciais de mortalidade por sexo. As mulheres costumam desfrutar de uma expectativa mais longa e menores níveis de mortalidade do que homens. O autor conjectura que, em países desenvolvidos, a expectativa de vida masculina pode ser maior e os níveis de mortalidade masculinos menores. Porém, apesar das hipóteses, deixou em aberto o motivo pelo qual a correlação entre TFTM e TFTF se modifica no ponto em que os países alcançam a TFTs no nível de reposição.

\begin{comment}
- diferenças de idade na parturição ---
\cite{dudel2021male}:
----Embora observemos padrões bastante diversos nas diferenças de idade entre países e grupos de países, o mecanismo demográfico subjacente é, em todos os casos, diferenças de gênero no adiamento. Especificamente, em todos os países que estudamos, a idade média ao dar à luz tem aumentado continuamente desde 1990, tanto para homens como para mulheres (ver materiais suplementares para obter detalhes); ou seja, tanto homens como mulheres têm adiado o parto. Em combinação com uma idade média de parto mais elevada para os homens, esta conclusão sugere que as mudanças na diferença de idade parental no momento do parto se devem a diferenças na velocidade com que o adiamento avança: se progredir mais rapidamente entre os homens do que entre as mulheres, a diferença de idade aumenta; mas se, por outro lado, avança mais rapidamente entre as mulheres do que entre os homens, a diferença de idade diminui----

tefs - United Nations Demographic Yearbook 2021




POP ESTÁVEL  (schoumaker)
    Em contraste, os países ocidentais combinam baixas diferenças de idade entre parceiros e estruturas etárias mais avançadas. Ambos os factores estão em jogo nas diferenças entre homens e TFRs femininas, como será demonstrado no caso de populações estáveis. 



IMIGRAÇÃO: (schoumaker)

    O efeito da migração internacional é visível em alguns países onde a razão é muito inferior ao esperado devido à diferença de idade (por exemplo, no Bahrein e no Qatar, com grande imigração masculina), ou superior ao esperado em países com grande emigração masculina, como no Nepal com o DHS. Em contraste, as estimativas que corrigem a migração internacional (seja no Nepal ou no Qatar) estão mais em linha com outras estimativas.

    (ZHANG)
When the cumulative pattern of fertility is considered, male total fertility rates (TFRs) are found to be different from those of females. It is shown that male TFRs were first higher than female TFRs in most Western industrialized countries before the 1960s. Such male and female fertility differentials are likely to be resulted from the relative shortage of men caused by two world wars. Since the 1960s, males in most industrialized countries have recovered from war time losses. Coupled with an increasing emigration that has been replaced by immigration which is largely dominated by men, male TFRs turned to be higher than female TFRs afterwards (Coleman, 2000). In addition to male and female fertility differentials in rates, demographers have also indicated that the progeny size distribution and childlessness patterns of men and women differentiate male fertility from female fertility.

\end{comment}


\begin{comment}
    A nossa segunda conclusão principal é que as diferenças de idade entre pais e mães permaneceram constantes ou diminuíram na maioria dos países do nosso estudo, exceto nos países da Europa Oriental e na Alemanha Oriental. Embora esta tendência geral esteja em linha com as expectativas baseadas nas teorias de gênero, as diferenças que existem atualmente entre os países parecem ser afetadas por mais factores do que apenas diferenças nos níveis de igualdade de gênero. Embora as nossas descobertas indiquem que a fecundidade masculina tem sido geralmente inferior à fecundidade feminina nos últimos anos, esta tendência não parece manter-se a nível mundial. 

    Schoumaker (2017, 2019) mostrou que em contextos em que a poliginia é praticada e as populações estão a crescer rapidamente, os níveis de fecundidade masculina podem por vezes ser duas vezes superiores aos níveis de fecundidade feminina. Seus resultados sugerem que esse padrão é impulsionado, em grande parte, pelas diferenças de idade entre os casais. 
    
    Esta conclusão está em linha com os resultados da nossa análise contrafactual, que indicam que os casos extremos tendem a desaparecer quando os pais são atribuídos à mesma coorte que as mães, e que grandes diferenças de gênero nos níveis de fecundidade são muitas vezes motivadas por diferenças no momento da fecundidade. . No outro extremo, o razão TFT mais baixo relatado na literatura é para Inglaterra e País de Gales em 1973, em torno de 0,89 (Schoen 1985). O valor que encontramos para a Alemanha Oriental, 0,84, está abaixo deste nível e pode indicar que os homens da Alemanha Oriental têm experimentado o que Schoen (1985) chamou de “aperto de natalidade”: ou seja, na Alemanha Oriental, o número desigual de homens e mulheres na faixa etária reprodutiva está tendo impacto na fecundidade dos homens. Geralmente, a variação nas razões TFR que observamos entre países e ao longo do tempo não é negligenciável. Os grupos de países com semelhanças culturais e/ou políticas parecem ter padrões de tendências mais semelhantes. Esta descoberta requer uma investigação mais aprofundada.




---- 
dudle 2016: 

as diferenças de idade dos pais aumentaram para as mães mais jovens e diminuíram para as mães mais velhas; e a heterogeneidade entre as faixas etárias aumentou. As inclinações para os homens apresentam menos variação do que as inclinações para as mulheres. 

na perspectiva dos pais, as diferenças de idade entre os pais têm geralmente diminuído, mas também tem havido um padrão bastante estável de pais mais velhos que tendem a ter filhos com mães muito mais jovens.

“observations that men generally prefer younger partners as they become older, while women’s age preferences for their partners are more heterogeneous (Skopek, Schmitz, and Blossfeld 2011).” ([Dudel et al., 2020, p. 13](zotero://select/library/items/XGG8CKCJ)) ([pdf](zotero://open-pdf/library/items/54XEELWP?page=14\&annotation=LPQ2UYNK))

a mudança nas diferenças de idade dos pais ao longo do tempo foi influenciada pelo género. Para as mulheres, esta mudança levou a uma maior heterogeneidade entre idades, de tal forma que as diferenças de idade dos pais entre mães mais jovens e mais velhas diferem mais hoje do que no passado. Entre os pais, por outro lado, as diferenças de idade parentais diminuíram, independentemente da idade em que tiveram um (outro) filho. Estas observações estão em linha com as conclusões existentes de que as alterações na fecundidade por estatuto social ou por idade que ocorreram nas últimas décadas foram muito maiores para as mulheres do que para os homens (ver Jalovaara et al. 2018). Uma possível explicação para isto é que o estatuto socioeconômico das mulheres tem vindo a mudar mais rapidamente. No entanto, embora seja claro que os padrões de disparidades de idade dos pais são de gênero, os resultados das nossas análises de regressão indicam que a diminuição das diferenças de idade dos pais que são atribuíveis ao atraso na parentalidade apresenta regularidades notáveis entre países ao longo do tempo.

indicam que, para as mulheres, o parto mais tardio está associado a uma menor diferença de idade parental; e, portanto, com uma relação de poder mais equilibrada. Nas últimas décadas, detectamos que as mães mais jovens têm filhos com parceiros cada vez mais velhos, o que sugere que é cada vez mais provável que estejam numa união com relações de poder desiguais. Para os pais, o parto mais tardio está associado a uma maior diferença de idade parental, o que implica que estes tenham maior poder na relação. No entanto, este efeito tem diminuído ao longo do tempo à medida que o adiamento da fecundidade aumentou entre homens e mulheres.” ([Dudel et al., 2020, p. 13]


\end{comment}
