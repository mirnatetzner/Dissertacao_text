\section{Fecundidade masculina pelo mundo}\label{tft_tef_mundo}

Cada vez mais há uma percepção de que é relevante compreender os aspectos reprodutivos masculinos, uma vez que a fecundidade masculina possui aspectos específicos, que a distingue da fecundidade feminina. \citeonline{zhang2010male} investigou os diferenciais nas TFTs e das TEFs femininas e masculinas utilizando como fonte principal a publicação \textit{“Demographic Yearbook 2001-Special Topic: Natality Statistics”}\cite{unstatsu25:online2001}. O autor analisou, ao todo, dados de 43 países entre os anos 1990 e 1998, utilizando o intervalo de idade (idade fértil) para as Taxas de Fecundidade Específicas por idade femininas (TEFF) de 15-49 e 15-59 para o cálculo das Taxas de Fecundidade Específicas por idade masculinas (TEFM). A partir de sua pesquisa, \citeonline{zhang2010male} constatou que o coeficiente de variação (CV) para as Taxas de Fecundidade Total Masculinas (TFTM) foi de 0,44, enquanto as Taxas de Fecundidade Total Femininas (TFTF) tiveram o CV de 0,37, indicando uma variação maior nos níveis das TFTMs em relação às TFTFs entre os países analisados. Em seu trabalho, o autor identifica ainda que, em países onde a fecundidade feminina é alta (estabelece TFTF > 2,2), a diferença entre a TFTM e TFTF tende a ser maior, enquanto países com TFTF abaixo desse valor tendem a apresentar taxas comparáveis ou mesmo um nível de TFTF levemente acima da TFTM. Suas conclusões indicam que as TFTMs estão mais espalhadas em relação à média, entre os países analisados, do que as taxas femininas. Ou seja, o número médio de filhos tidos pelas mulheres ao longo de sua vida reprodutiva (nos países analisados) é mais concentrado em torno da média feminina do que o número de filhos tidos por homens em relação a sua média. Adicionalmente, a transição de regimes de alta fecundidade para níveis mais baixos parece influenciar a relação entre fecundidade feminina e  masculina.


 Um segundo trabalho, referência para a área da FM, realiza uma revisão a partir de dados para a idade do pai de 163 países no período por volta de 2011. Nele, \citeonline{schoumaker2019male}, corroborando o que foi encontrado por \citeonline{zhang2010male}, observa que a medida que uma população passa pela transição demográfica, a fecundidade masculina e a feminina se aproximam. O número médio de filhos tidos por mulher variou de 1 a 8 filhos entre os países analisados, enquanto para homens o intervalo encontrado foi bem maior. As TFTMs encontradas pelo autor \footnote{O autor calcula a partir das Taxas de Fecundidade Específicas por idade (TEFs), com 7 grupos etários para mulheres, entre 15 e 49 anos e para homens 13 grupos, entre  15 e 79 anos. } para países da África Subsaariana, foram frequentemente 1,5 a 2 vezes maiores que as TFTFs, combinado com uma população mais jovem e diferenças de idade ao nascimento entre homens e mulheres próximas ou superiores a 10 anos. Enquanto nos países ocidentais, com estruturas etárias mais avançadas, a diferença entre homens e mulheres se mostrou menor, tanto na idade ao nascimento, entre dois e quatro anos, como nas TFTs, com uma razão da TFTM pela TFTF, muitas vezes, inferior a um, ou seja, com a TFTF maior do que a TFTM.

\citeonline{schoumaker2019male} encontra valores mais baixos para a TFTM  nos países europeus, em média entre um e dois filhos, sendo geralmente próximos da TFTF. Os países asiáticos, por outro lado, apresentaram uma variedade maior das TFTMs, com o Japão e a Coreia do Sul apresentando cerca de 1,2 filhos por homem, enquanto outros países do continente apresentam valores bem mais altos, Paquistão (5,1), o Iraque (5,5) e o Afeganistão (6,9). Para os países da América Central e do Sul os valores encontrados foram, no geral, baixos, com o Chile, a Costa Rica e Cuba, variando entre menos de dois filhos, apesar de haver o Haiti como exceção, com TFTM superior a cinco filhos por homem. As TFTMs com valores mais elevados, encontrados pelo autor, se concentraram na África Subsariana, onde, dos 43 países analisados (para essa região), metade possuia uma TFTM superior a 8,5 filhos e um quarto dos países, apresentaram um valor superior a 10 crianças por homem. Seu estudo é eficiente em ilustrar a variabilidade que a fecundidade masculina pode alcançar em diferentes contextos.

Outra contribuição de \citeonline{schoumaker2019male} está em observar que países com alta proporção de mulheres em relações poligínicas\footnote{Estado de um homem que está casado com muitas mulheres. “poliginia”, in Dicionário Priberam da Língua Portuguesa [em linha 2], 2008-2024, https://dicionario.priberam.org/poliginia.} estão relacionados a altos valores para a razão da TFTM pela TFTF. Porém, apesar de ser relevante, a poliginia por si só não se mostrou ser fator explicativo determinante, como o autor chama a atenção, essa diferença costuma estar associada a uma condição necessária à poliginia, a diferença de idade entre os cônjuges numa população com crescimento positivo \apud{pison1986demographic}{schoumaker2019male}.

Nesse sentido, a diferença de idade entre parceiros parece ser um fator importante para os diferenciais na TFT de homens e mulheres. Em trabalho recente, \citeonline{dudel2021male} elencam três possíveis razões para as diferenças encontradas nos valores das taxas de fecundidade masculinas e femininas: 1 - Diferenças de idade entre mães e pais combinado com variação nos tamanhos das coortes; 2 - Diferenças no tamanho das populações feminina e masculina; 3 - Diferenças de gênero no efeito tempo.  

O efeito tempo é uma distorção que pode ocorrer em medidas transversais de fecundidade, como é o caso da TFT. Uma taxa transversal é aquela que mistura eventos de diferentes coortes (ao contrário de taxas que acompanham coortes de maneira longitudinal)\cite{FOZ2021metodos}. Isso faz com que a TFT seja afetada pela experiência de coortes distintas, sendo sensível a mudanças comportamentais temporárias, flutuações de curto prazo que pode ser resultado de um adiamento ou adiantamento momentâneo da fecundidade.

Assim, \citeonline{dudel2021male} propõem que a diferença entre as TFTs masculina e feminina poderia advir de diferenças entre os sexos no  adiamento ou adiantamento da maternidade - paternidade. Segundo os autores, apesar de se observar um padrão geral de pais mais velhos que mães, em diferentes momentos e contextos sociais, alguns fatores, como a revolução de gênero, o aumento do nível educacional entre mulheres, costumam estar associados a um adiamento da maternidade e podem criar condições propícias para a diminuição do intervalo de idade entre parceiros (e.g. pela diminuição da dependência das mulheres em relação à renda do cônjuge).

Embora o adiamento da maternidade seja um tema intensivamente estudado pela demografia no contexto da transição demográfica, relativamente pouca atenção tem sido destinada aos fatores associados ao adiamento da paternidade, com exceção a alguns autores \cite{kyzlinkova2018fatherhood,olah2008sweden,office2009patterns,DenmarkNordfalk} que, no geral, têm mostrado o efeito tempo para os homens, da postergação da idade média ao nascimento, tende a acompanhar o das mulheres, porém muitas vezes se mostrando menos atenuado\footnote{\citeonline{DenmarkNordfalk} encontra um efeito tempo negativo para homens e mulheres dinamarqueses entre 1980-2010, com o efeito para as mulheres, mais evidente do que para os homens}. \citeonline{dudel2021male} chegam a conclusão de que as diferenças no adiamento da idade média ao nascimento entre os sexos e a diferença de idade entre parceiros são fatores relevantes para os diferenciais nas TFTs entre os sexos, apontando para tópicos de pesquisa vitais para o campo da FM. 

No que diz respeito às diferenças no tamanho das populações feminina e masculina, ocorre impacto sobre o denominador no cálculo das TEFs, a população exposta. Por exemplo, se a proporção de homens emigrantes de determinado país é maior que a de mulheres emigrantes, a população masculina diminui, podendo temporariamente aumentar a taxa de fecundidade masculina em relação à feminina. 



\begin{comment}

A combinação de diferentes idades entre mães e pais com a variação nos tamanhos das coortes produz discrepâncias na fecundidade de homens e mulheres porque, se homens e mulheres forem de coortes diferentes e houver variação no tamanho das coortes, o denominador das taxas masculinas e femininas podem diferir. \citeonline{schoumaker2017measuring}, por exemplo, demonstrou em países com altas taxas de crescimento populacional a TFTM cresce mais do que a TFTF, já que os pais costumam vir de coortes menores e de idade mais avançadas do que suas parceiras. 




   -----------------------
     A nossa segunda conclusão principal é que as diferenças de idade entre pais e mães permaneceram constantes ou diminuíram na maioria dos países do nosso estudo, exceto nos países da Europa Oriental e na Alemanha Oriental. Embora esta tendência geral esteja em linha com as expectativas baseadas nas teorias de género, as diferenças que existem actualmente entre os países parecem ser afectadas por mais factores do que apenas diferenças nos níveis de igualdade de género.
    -------------- 


Razão de sexo\footnote{Expressa pelo número de homens para cada grupo de
100 mulheres, na população residente em determinado espaço geográfico, no ano
considerado \cite{rede2002indicadores}.}


encontraram indícios de que as mudanças na razão entre a TFTM e a TFTF são governadas pelas diferenças de idade entre parceiros e em mudanças no comportamento reprodutivo, como adiamento da paternidade e maternidade. Analisaram dados de 17 países de renda alta, no intervalo entre 1968 a 2016 ou 1980 a 2016, para a maioria dos países, e chegaram a algumas conclusões interessantes. Segundo os autores,\textbf{o adiamento da maternidade e da paternidade são fatores...}


From a demographic perspective, such disparities can be driven by three main factors: first, differences in the population sizes of males and females, i.e., in the sex ratio, which can affect both TFR and CFR differences; second, age differences between mothers and fathers in interplay with variation in cohort sizes (affecting the TFR and the CFR); and, third, gender differences in tempo effects, which can impact gender differences in the TFR, but not in the CFR.



---
\textbf{quando os pais são atribuídos à mesma coorte que as mães, e que grandes diferenças de gênero nos níveis de fecundidade são muitas vezes motivadas por diferenças no momento da fecundidade. }


EM WONG 2020- SOBRE SCHOUMAKER 2019:

    The author shows that the TFR of men is almost always higher than that o

women, which he explains by two factors. The first is that men have a higher mortality rate than women, which would lead to fewer men exposed to the risk of parenthood. The second is that men tend to bond and have children with women younger than themselves. If these cohorts of men and women were born in a context of positive population growth, then the cohort of men tends to be smaller than that of women, also decreasing the population at risk of paternity.
\end{comment}
