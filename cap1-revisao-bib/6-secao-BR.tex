\section{Fecundidade masculina no contexto brasileiro}

A fecundidade, em comparação com mortalidade, é um componente da dinâmica demográfica fortemente afetado pelo contexto social. Numa perspectiva de estrutura e agência, as circunstâncias históricas e o território compõem o quadro de ações possíveis dos sujeitos, dos constrangimentos estruturais que guiarão a trajetória de vida de um indivíduo em determinada sociedade. Nesse sentido, a fecundidade masculina e as preferências reprodutivas podem apresentar determinantes específicos guiados pelos significados da construção da paternidade no Brasil e na América Latina.

\citeonline{giffin1999homens} analisando uma série de estudos (majoritariamente qualitativos) das décadas de 1980 e 1990, mostram que a fala masculina sobre sexualidade e afeto e da relação de homem-mulher é condicionada por um padrão histórico que ressalta a hierarquia dos gêneros e a desvalorização desses assuntos, vistos como femininos. De fato, o papel mais próximo que os homens parecem ocupar nas análises sobre aspectos reprodutivos parece ser sobre a sua influência enquanto cônjuge na utilização de contraceptivos de suas parceiras \cite{carvalho2001participaccao}. Pensar sobre os processos reprodutivos masculinos e seu impacto nas dinâmicas demográficas e sociais é essencial para identificar fatores que tem impactado na redução da fecundidade no país, principalmente no contexto das mudanças sociodemográficas recentes, tais como taxas mais elevadas de divórcio, aumento da participação da mulher no mercado de trabalho, transformações na organização do domicílio, urbanização, que passaram a demandar maior envolvimento dos homens na criação dos filhos. 

Em consistente decrescimento desde a década de 1960, a TFT brasileira atingiu, já no censo de 2010, patamar abaixo daquele do necessário à reposição populacional (2,1 filhos nascidos vivos por mulher). Segundo projeção do \textit{“World Population Prospects 2022”}\cite{desa2022united}, o total populacional brasileiro deve começar a declinar já próximo do início da década de 2050. Tais mudanças no cenário colocam em foco o estudo da fecundidade na demografia brasileira, de maneira geral. A transição da fecundidade brasileira é um fenômeno que está ocorrendo rapidamente, passando de um país com cerca 6,1 filhos por mulher na década de 1950 para o cenário atual. 

A fecundidade, também no Brasil, é abordada, quase que de maneira universal, como um fenômeno feminino, com seus determinantes e seus diferenciais segundo as variáveis demográficas (cor-raça, região, nível educacional, etc.) sendo, ainda que não exaustivamente, bem documentados\footnote{Para citar alguns exemplos: \cite{gonccalves2019transiccao,berquo2014notas,wong1984niveis,wong1983fecundidade}}. A fecundidade masculina, por outro lado, tem se mantido como uma área em que poucos pesquisadores ousaram adentrar. Alguns trabalhos \cite{gerber2014two, Caswell2022-zu}, porém, têm demonstrado a importância de considerar também a fecundidade masculina nas principais análises demográficas.

O trabalho de \apudonline{Wong1986}{wong2022fecundidade} foi pioneiro no Brasil nos estudos da FM. As autoras realizaram algumas análises utilizando dados administrativos\footnote{Como veremos no capítulo \ref{Sinasc&DNV}, à época o Sinasc ainda não havia sido criado.} do estado de São Paulo no ano de 1983. Seus resultados demonstraram (como visto na seção anterior) que, no ano analisado, em cerca de 10\% dos registros de nascimento de SP a idade do pai aparecia como “ignorada”. As autoras também observaram que a fecundidade masculina para o estado atingiu seu nível máximo entre os homens que tinham entre 20 e 30 anos e que, entre as mulheres, esse nível dava-se cerca de 5 anos antes, corroborando resultados encontrados na literatura que mostraram que o pico da fecundidade masculina é posterior ao feminino.

No que foi uma das primeiras pesquisas a investigar aspectos reprodutivos masculinos nacionalmente, a PNDS 1996 entrevistou homens entre 15 e 59 anos. Na época, a pesquisa encontrou que 87\% dos homens unidos declararam ter pelo menos um filho. 56\% dos homens relataram nunca terem tido filhos, 27\% ter um ou dois filhos e 17\% pelo menos três\apud{badiani1998homens}{falcao_fecundidade_2013}. A diferença na idade mediana ao casar entre homens e mulheres era em torno de três anos, enquanto a idade mediana na primeira relação para homens era de 16,7 anos mostrando que eles começavam a vida sexual aproximadamente 2,8 anos mais cedo do que as mulheres, de maneira oposta ao que acontece com o casamento, em que o homem casaria três anos mais tarde\cite[Tabela 5.7]{bemfam_brasil_1997}.

Mais recentemente, \citeonline{wong2020LatinA} traçaram um panorama da TFT masculina entre os anos de 1970 e 2010, mediante métodos indiretos aplicados a dados censitários disponíveis na plataforma do \textit{IPUMS International}\footnote{Citado na seção anterior (\ref{bases_e_metodos-ref})}. Os autores encontraram que, no Brasil,
o diferencial\footnote{(TFTM - TFTF)} entre a TFTM e a TFTF passou de, aproximadamente, 1,3 em 1970, para 0,2 filhos em 2010. A TFTM caiu em média 1,06 a cada década, tendo o valor inicial, em 1970, de 6,2 filhos por homem e valor final de 2,0, em 2010. Enquanto isso, a TFTF encontrada pelo método dos autores foi de 5,0 no mesmo período inicial e 1,8 em 2010, reduzindo 0,79 a cada 10 anos. Calculando a idade média da fecundidade de homens e mulheres, encontraram que, na década de 70, o valor para a população masculina era de 36,2 e para mulheres 29,8 anos, já no final do período, em 2010, os valores foram de 32,5 e 26,6, respectivamente. A diferença entre sexos para a idade média da fecundidade apresentou pouca variação, sendo de 6,4 em 1970 e 6,0, em 2010.

A análise realizada no trabalho de \citeonline{wong2020LatinA} segue o padrão observado em outros países, onde a diferença entre as TFTs masculina e feminina diminuem a medida que o país passa pela transição demográfica. A TFT masculina decresceu em ritmo mais acelerado que a feminina, com valores muito próximos ao final do período analisado. Apesar disso, a diferença da idade média da fecundidade permaneceu alta. Os autores argumentam que a queda (para ambos os sexos) do indicador pode ser parcialmente explicado devido ao crescimento da fecundidade adolescente na América Latina durante o intervalo estudado. No Brasil, a tendência alcançou o nível mais baixo para homens e mulheres na década de 2000, com uma leve subida em 2010. Caso seja mantida a tendência de diminuição da fecundidade adolescente e ocorra um aumento na proporção das mulheres que postergam a maternidade, é esperado encontrar um aumento da idade média da fecundidade feminina, padrão que, segundo os autores, pode ser refletido na fecundidade masculina. 

Trazendo mais uma contribuição relevante, o artigo decompõe\footnote{Referenciam o método a Kitagawa (1949).} as mudanças na idade média da fecundidade masculina em dois efeitos, o efeito das mudanças no padrão de nupcialidade e nos padrões da fecundidade feminina. O primeiro mede o papel das mudanças da diferença de idade entre casais, enquanto o segundo mensura o impacto da mudança de estrutura de idade da fecundidade feminina sobre a estrutura de idade na fecundidade masculina. Ou seja, um efeito indica qual seria o impacto do adiamento da fecundidade na idade média da fecundidade masculina, caso o intervalo de idade dos casais fosse constante, o segundo sinaliza o efeito das mulheres terem filhos com companheiros mais jovens. Assim, afirmam que, ao mesmo tempo que a fecundidade feminina está diminuindo, as mulheres estão tendo filhos com homens mais jovens do que no passado. A partir da separação dos dois efeitos, concluem que a mudança no padrão de nupcialidade brasileiro levou ao rejuvenescimento da fecundidade masculina, independentemente das mudanças na estrutura da fecundidade feminina que também foram experimentadas nas últimas décadas. Em sua conclusão, \citeonline{wong2020LatinA} apontam para a necessidade de melhor compreender os padrões de idade para nupcialidade nos países da América Latina, calculando medidas de fecundidade masculina para subgrupos populacionais, de modo a compreender o rejuvenescimento da fecundidade masculina para além do efeito da redução da idade média da fecundidade feminina.

Em 2019 foi a campo a Pesquisa Nacional de Saúde (PNS) que entrevistou indivíduos de 15 anos ou mais, coletando informações referentes à paternidade e pré-natal do parceiro, dentre os quesitos investigados, o número de filhos nascidos vivos declarados diretamente pelos homens. \citeonline{wong2022fecundidade} utilizam dados da pesquisa para estimar a parturição masculina (ou número de filhos tidos por idade) e avaliar o impacto de variáveis selecionadas sobre a fecundidade masculina. Descrevem que a fecundidade masculina no Brasil, assim como a feminina, apresenta diferenças entre as grandes regiões, com o Norte e o Nordeste apresentando um número médio de filhos mais elevado para homens e mulheres, em comparação com as regiões Sul e Sudeste. O comportamento da FM também apresenta diferenças em relação à raça-cor, de forma similar às mulheres, homens pretos e pardos apresentam um número médio de filhos mais elevado que homens brancos. Assim também ocorre com o nível de instrução, associado inversamente ao número médio de filhos.  

Ainda segundo relatório produzido pelo IBGE: “Pesquisa Nacional de Saúde 2019 – Ciclos de vida”, no ano do levantamento, cerca de 64,6\% dos homens brasileiros já haviam sido pais de pelo menos um filho ou filha (que em 2019, tinham 15 anos ou mais de idade). Entre os jovens de 15 a 29 anos, 19,0\% eram pais, enquanto na faixa de 30 a 39 anos esse percentual foi de 68,9\%, entre 40 a 59 anos foi 85,3\% e com 60 anos ou mais alcança 91,4\% \cite{instituto2021pesquisa}. Cabe lembrar que as faixas mais jovens estão no início do ciclo reprodutivo e podem ser afetadas, por exemplo, por efeitos de postergação da paternidade, como ocorre em outros países \cite{kyzlinkova2018fatherhood, dudel2020unexplored}. Segundo \citeonline{instituto2021pesquisa}, a idade média no momento do nascimento do primeiro filho entre os homens de 15 anos ou mais que já tinham filhos foi de 25,8 anos, onde na área urbana esse valor foi de 26 anos e de 24,9 anos na área rural. Os homens da Região Norte foram os que tiveram o primeiro filho mais cedo, com 24,3 anos e os do Sudeste os que tiveram mais tarde, em média 26,6 anos. 

\begin{comment}
    ESSE ARTIGO PODE SER MAIS EXPLORADO: \citeonline{wong2022fecundidade}

    DADOS DO RELATÓRIO: Pesquisa Nacional de Saúde (pns) 2019 – Ciclos de vida

    
    \cite{wong_male_nodate,franco_agora_nodate,de_carvalho_fecundidade_nodate, carvalho_apoio_2000, wong_fecundidade_nodate, falcao_fecundidade_2013, rutenberg_instituto_nodate, oliveira1999homens,siqueira_saue_2000, franco_os_nodate}

Scoppetta, O. Cambios en las trayectorias de fecundidad masculina en Córdoba, Colombia. Papeles de población, n. 15, v. 62, p. 173-199, 2009.

\end{comment}

Entretanto, se as pesquisas envolvendo o estudo da FM a partir de pesquisas domiciliares e censos no Brasil já são escassas, quando restringimos para estudos da temática que utilizam como fonte dados administrativos no Brasil, esse campo fica ainda mais restrito. Além do trabalho de \citeonline{Wong1986}, ao qual foi possível ter acesso graças ao artigo de \cite{wong2022fecundidade}, o artigo de \citeonline{falcao_fecundidade_2013} parece ser um dos poucos que se arriscaram em utilizar microdados do Sistema de Informação sobre Nascidos Vivos. Ainda assim, como explicado na seção anterior(\ref{bases_e_metodos-ref}), o autor se restringe a analisar municípios de São Paulo onde os registros com idade do pai não declarada eram, no máximo, 8,0\%. Analisando dados de 14 municípios paulistas para o ano de 2013, encontrou TFTs próximas para os dois sexos, com a TFTM levemente mais alta em 12 dos 14 municípios. Em 13 deles o pico da fecundidade masculina ocorreu na faixa de 30 a 34 anos (nível mais alto da TEFM). Foi possível identificar o diferencial de idade de pais e mães entre parceiros, com homens sendo mais velhos na maior parte dos grupos etários. 

\citeonline{wong2022fecundidade} enfatizam o papel fundamental do estudo da fecundidade masculina na transição demográfica não só no Brasil, mas no contexto dos países latino-americanos. Segundo os autores, o aumento da longevidade e o surgimento de novos arranjos familiares ao longo dos ciclos de vida são fenômenos que evidenciam a necessidade de ampliar o número de estudos que investiguem as múltiplas questões envolvidas com a fecundidade dos homens. Entretanto, o desafio que se impõe para os estudos da fecundidade masculina no Brasil vão além da superação do paradigma da abordagem de um sexo (feminino) e incluem a melhoria dos sistemas de registros de nascimento, ainda mais no que se refere à conscientização sobre a importância do preenchimento das informações relacionadas à idade do pai nesses registros. 
 
Uma vantagem de estimar a fecundidade masculina a partir de registros administrativos é a cobertura da base. Abastecida por documento obrigatório para todo brasileiro nascido vivo, teoricamente possibilitam abarcar informações para todos os nascidos vivos (principalmente a partir das melhorias quanto a cobertura, como veremos adiante no Capítulo  \ref{Sinasc&DNV}). Com relação à periodicidade de disponibilização dos dados, que é restringida apenas pelo período de consolidação das informações, ao contrário das pesquisas domiciliares e censos que costumam ter um intervalo bem maior. Uma desvantagem, evidentemente, é a proporção de dados faltantes que necessitam de tratamento e que, caso não sejam tratados de maneira correta, pode levar a viés nas estimativas. Complementarmente, um desafio que se impõe é que a base é composta principalmente por informações da parturiente e do recém-nascido, onde as únicas informações coletadas do pai, são: nome (não disponibilizado) e idade. Porém, por vezes, dados da mãe são utilizados como proxy para o dado do pai, como em \citeonline{dudel2016estimating}, no qual assumem que os pais vivem na mesma região que a mãe. 

Como visto anteriormente, alguns autores abordaram a questão sobre dados faltantes para informação da idade do pai empregando técnicas de imputação condicional à idade da mãe, tanto na utilização de registros administrativos \cite{dudel2019estimating} como de pesquisas domiciliares \cite{schoumaker2017measuring} e censos \cite{wong2020LatinA}. Não foi possível encontrar na literatura, imputação da idade do pai condicionada por outra variável. 

Espera-se identificar, através do trabalho aqui proposto, diferenciais das taxas masculinas e femininas de fecundidade no Brasil para os anos de 2012 a 2022, utilizando métodos de imputação para a variável relativa à idade do pai, presente na base do Sinasc e alimentada a partir da declaração de nascido vivo. E, dessa forma, contribuir para a literatura da fecundidade masculina brasileira a partir do desenvolvimento de métodos adequados para solucionar o principal empecilho para a utilização da base do Datasus na mensuração taxas de fecundidade masculinas, i.e. a alta proporção de dados faltantes para a variável idade do pai.  

\textcolor{red}{[complementar com dados para América Latina]}