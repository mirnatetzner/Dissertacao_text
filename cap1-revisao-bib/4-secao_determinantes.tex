\begin{comment}

FATORES SOCIOLÓGICOS, ESTUDOS DE GÊNERO E PREFERENCIAS REPRODUTIVAS ENTRE HOMENS E MULHERES: 
\begin{itemize}
    \item Challenging Demography: Contributions from
Feminist Theory (Nancy E. Riley)
    \item Masculinity: An Overlooked Cultural Influence on Fertility
Rachael S. Pierotti, University of Michigan
\end{itemize}


SOCIOLOGICO (zhang) 
    Beyond investigating fertility determinants, examining other dimensions of fertility needs involving males as well. For example, understanding the timing of parenthood demands exploring closely the meaning of fatherhood and motherhood in various cultural institutions and how the meaning changes over the life course for men as compared to for women. Looking into the link between the construction and deconstruction of a childbearing and childrearing union (such as cohabitation and marriage) also requires knowing more about men’s commitment in these unions. The women’s health movement has also stressed the need for men to be aware of their responsibilities in family planning and reproductive health. In sum, it is imperative to bringing men into fertility and fertility-related research. Male fertility should be considered as an important component of fertility studies.


https://citeseerx.ist.psu.edu/document?repid=rep1&type=pdf&doi=db7234060968eb15b6239c3661d5dcac5159c3a5 
    Este artigo demonstra que a educação influencia o comportamento reprodutivo dos homens de múltiplas maneiras. Centrando-nos particularmente na falta de filhos e na fecundidade multiparceira, os elementos-chave nas nossas análises são factores relacionados com a capacidade de um homem para uma parentalidade económica e prática, reflectida, por exemplo, na sua capacidade parental. através das perspectivas de rendimento, da segurança no emprego, da flexibilidade laboral e da composição de género do emprego.

\citeonline{goldscheider1996fertility}





- estudos atuais(relevancia atual):  \cite{keilman2014measures}, \cite{gerber2014two}


\section{referencias do Zhang -- antigas:}

\begin{itemize}
  
    \item Os efeitos da idade na fecundidade masculina e feminina também são muito diferentes. As mulheres atingem seu pico de fecundidade entre 25 e 35 anos, após o qual a fecundidade começa a declinar até a menopausa. Já os homens continuam a ter capacidade reprodutiva até a morte, embora de forma gradual (Wood, 1994). 
    \item Quando o padrão cumulativo de fecundidade é considerado, as taxas de fecundidade total masculina (TFR) são encontradas como diferentes das femininas. Observa-se que as TFR masculinas eram maiores que as femininas na maioria dos países industrializados ocidentais antes dos anos 1960.
        \begin{itemize}
            {\item Essas diferenças de fecundidade entre homens e mulheres provavelmente resultaram da relativa escassez de homens causada pelas duas guerras mundiais. Desde os anos 1960, os homens na maioria dos países industrializados recuperaram-se das perdas da guerra. Aliado ao aumento da emigração substituída pela imigração predominantemente masculina, as TFR masculinas passaram a ser maiores que as femininas (Coleman, 2000)}
       \end{itemize}
    \item Além das diferenças nas taxas de fecundidade entre homens e mulheres, demógrafos também indicaram que a distribuição do tamanho da prole e os padrões de ausência de filhos diferenciam a fecundidade masculina da feminina. Algumas pesquisas mostram que, em geral, os homens têm menos filhos que as mulheres. 
    Consequentemente, há percentagens mais altas de homens sem filhos em comparação com as mulheres (Coleman, 2000). Todos esses fatos sugerem que a fecundidade humana não pode ser totalmente representada pela fecundidade feminina. Usar as taxas de fecundidade feminina para representar as masculinas pode ser problemático, especialmente em sociedades onde as taxas de divórcio, novo casamento e migração são bastante altas. Pesquisadores sugerem que mesmo aplicar as taxas de fecundidade feminina marital para representar a fecundidade humana como uma solução alternativa é inadequado, pois as taxas de nascimentos não maritais em algumas sociedades são comparativamente altas e os homens têm mais probabilidade de se casar novamente após o divórcio do que as mulheres, o que torna a fecundidade masculina e feminina não comparável (Greene \& Biddlecom, 2000; Juby \& Bourdais, 1998; Magnani, Bertrand, Makani, \& McDonald, 1995).

    \item Além de investigar os determinantes da fecundidade, examinar outras dimensões da fecundidade também exige envolver os homens. Por exemplo, entender o momento da paternidade exige explorar de perto o significado da paternidade e maternidade em várias instituições culturais e como o significado muda ao longo da vida para homens em comparação com mulheres. Analisar a ligação entre a construção e desconstrução de uma união de criação de filhos (como coabitação e casamento) também requer saber mais sobre o compromisso dos homens nessas uniões. O movimento pela saúde da mulher também destacou a necessidade de os homens estarem cientes de suas responsabilidades no planejamento familiar e na saúde reprodutiva. Em suma, é imperativo incluir os homens nas pesquisas de fecundidade e relacionadas à fecundidade. A fecundidade masculina deve ser considerada um componente importante dos estudos de fecundidade.
\end{itemize}






-----------
\end{comment}