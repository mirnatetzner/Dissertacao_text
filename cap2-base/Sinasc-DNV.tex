\section{O Sinasc e a Declaração de Nascido Vivo}

Até a década de 1990 no Brasil, os nascimentos eram registrados no Sistema de Registro Civil. Portanto, se tinha conhecimento apenas dos nascimentos informados em cartório, o que ocasionava um sub-registro \cite{silvestrin2018avaliaccao}. As informações coletadas referiam-se, basicamente, à comprovação legal do evento, sem prover dados importantes para a área de saúde, como as condições da criança ao nascer \cite{szwarcwald2019evaluation}. Em 1989, um grupo assessor foi criado pelo Ministério da Saúde, visando ampliar, reformular e aprimorar o processo de produção e disseminação das Estatísticas Vitais (Gevims), no contexto da informatização dos serviços de saúde e redemocratização do país. O grupo atuou fomentando o debate entre órgãos estaduais responsáveis pela coleta e pela produção de dados da necessidade de implantar um sistema de informação sobre nascidos vivos, o Sinasc. Experiências internacionais e diagnósticos da condição interna dos registros de nascimentos (principalmente da ausência de informações relevantes sobre a saúde dos nascidos vivos) deram base para a necessidade da implementação do sistema e de seu documento base, a Declaração de Nascido Vivo (DNV) \cite{MSsinasc2009}. Nesse contexto, o Sistema de Informações sobre Nascidos Vivos foi implantado pelo Ministério da Saúde em 1990, com a intenção de subsidiar as intervenções relacionadas à saúde da mulher e da criança para todos os níveis do Sistema Único de Saúde (SUS). 

Por meio de diálogo com representantes de todas as unidades da federação foram selecionadas as variáveis para compor a DNV. Alguns dos critérios elencados foram que o documento deveria incluir um número reduzido de variáveis, porém, ao mesmo tempo, contemplar a diversidade regional de serviços de saúde em todo território nacional. Assim, o documento deveria ser adequado para o preenchimento, principalmente, dos profissionais de saúde e, ao mesmo tempo, suprir as necessidades dos gestores de saúde em diferentes níveis de desagregação \cite{MSsinasc2009}. Apesar da imposição legal para a efetivação do fluxo do Sinasc e para a implementação de seu documento base (a DNV) só ocorrer em julho de 1990, impulsionado pela promulgação do Estatuto da Criança e do Adolescente \cite{Lei:8.069:1990}, o sistema foi implementado prontamente pelas Secretarias Estaduais de Saúde \cite{BRMSlegislacao2009}. Ainda assim, devido à vasta extensão territorial brasileira, o desenvolvimento do Sinasc ocorreu de maneira heterogênea no país, fazendo com que os dados do sistema de informação passassem a ser divulgados apenas a partir de 1994 \cite{silvestrin2018avaliaccao}.

Como mencionado, o sistema é alimentado pela DNV, documento obrigatório para a lavratura da Certidão de Nascimento pelos Cartórios do Registro Civil, cujo modelo padrão passa a ser adotado nacionalmente a partir de 1990. A DNV, por definição, notifica o evento vital nascimento, para os nascidos vivos. Quando o produto da gestação não apresenta sinais vitais após a extração, é considerado morto, e apenas seu  óbito é notificado e, em caso de gravidez múltipla, deve ser preenchida uma DNV para cada produto da gestação, ou seja, para cada nascido vivo. Em caso de gestação por substituição (conhecido popularmente por barriga de aluguel) ou de adoção, deve ser considerada a informação da parturiente para preenchimento da DNV, ou seja, a pessoa que gerou e pariu a criança \cite{BRmanualDNV2022}.

Atualmente as unidades notificadoras reconhecidas pelo ministério da saúde\footnote{Art. 13, § 8º, da Portaria nº 116/2009} aptas a receberem formulários de DNV são os estabelecimentos e serviços de saúde (inclusive atendimento e internação domiciliar), profissionais de saúde e parteiras tradicionais vinculadas à unidade de saúde e os cartórios de Registro Civil, que poderão emitir a declaração para nascimentos que tiverem ocorrido a menos de três anos e tenham sido sem assistência de profissionais de saúde ou parteiras \cite{BRmanualDNV2022}. A declaração recebe três vias, a primeira é encaminhada para a Secretaria Municipal de Saúde, a segunda fica com o Cartório de Registro Civil, onde é arquivada no momento da emissão do registro de nascimento, e a última é arquivada no estabelecimento de saúde junto ao prontuário do médico do recém-nascido. 

Segundo o manual mais recente \cite{BRmanualDNV2022}, as DNVs têm como principal fonte as informações prestadas pela(o) parturiente e pelos profissionais de saúde presentes na sala de parto, porém deve-se utilizar também documentos disponíveis, como prontuários, Caderneta da Gestante e anotações pertinentes para auxílio do preenchimento. \citeonline{costa2009avaliaccao} apontam que a DNV pode ser preenchida, nos partos hospitalares, por uma variedade de profissionais de saúde. Entre obstetras, pediatras, enfermeiros-assistentes da sala de parto, estagiários, auxiliares de enfermagem, funcionários administrativos, entre outros, muitos dos quais não estarão necessariamente qualificados para a função. 

Muitos fatores podem afetar a qualidade das informações contidas no Sinasc \cite{bonilha2018cobertura}. Desde problemas no treinamento dos profissionais responsáveis pelo preenchimento que podem afetar a qualidade dos dados (problemas de incompletude e validade dos registros), da disposição das pessoas e instituições em participar e conduzir o sistema, ou questões relacionadas ao funcionamento do Sinasc, nível de cobertura, rapidez na inclusão do nascimento.

A avaliação de cobertura do Sinasc, diz respeito ao quanto dos nascimentos que ocorrem no país tem sido realmente registrados no sistema, e pode ser avaliada utilizando a razão entre Nascidos Vivos Informados e Estimados (via estatísticas oficiais produzidas no Censo Demográfico, Contagem Intercensitária, Pesquisa Nacional por Amostra de Domicílios, estimativas e projeções demográficas), onde valores abaixo de 100 indicam sub-registro do Sinasc \cite{rede2002indicadores}. Alguns autores demonstram como esse aspecto foi melhorando através do tempo \cite{gabriel2014evaluation, bonilha2018coverage}. Estima-se que na década de 1980, antes da implementação do Sinasc, o sub-registro do evento era da ordem de 22,6\% no país \cite{IBGEVitais1980}. Em 1994, ano em que passam a ser divulgados os dados do Sinasc, a região Norte e Nordeste apresentaram os piores valores para a Razão de nascidos vivos informados e estimados, de 65,5\% e 54,9\%, respectivamente \cite{rede2002indicadores}. Porém, o sistema foi se consolidando, em 2004 o Brasil apresentava uma cobertura de 89,4\%. Mais recentemente tem sido testada uma metodologia chamada “Técnica de Captura-Recaptura”\footnote{\href{https://biblioteca.ibge.gov.br/visualizacao/livros/liv101927.pdf#page=11.45}{IBGE. Estudo Complementar à Aplicação da Técnica de Captura-Recaptura Estimativas desagregadas dos totais de nascidos vivos e óbitos 2016 - 2019. Rio de Janeiro, 2022.}}, onde é feito o pareamento dos nascimentos registrados no Sinasc e aqueles contabilizados nas Estatísticas do Registro Civil, com o intuito de avaliar a cobertura do sistema. Ainda que seja uma estatística experimental\footnote{i.e. que estão sob avaliação porque ainda não atingiram um grau completo de maturidade em termos de harmonização, cobertura ou metodologia.}, é possível observar dos nascidos vivos no Brasil no ano de 2022, a proporção de subnotificação foi de 0,45\%, para partos realizados em hospitais, 1,01\% para partos realizados em outro estabelecimento de saúde sem internação, 4,07\% em partos domiciliares e 4,30\% na categoria  “outros” \footnote{\href{https://www.ibge.gov.br/estatisticas/sociais/populacao/26176-estimativa-do-sub-registro.html?edicao=32265\&t=resultados}{IBGE - Sistema de Estatísticas Vitais: Tabela 1.2}}. Demonstrando que, mesmo em partos realizados fora do ambiente hospitalar, o sistema parece estar próximo de uma cobertura completa.

A acurácia diz respeito à confiabilidade ou validade dos dados registrados. Segundo \citeonline{MSsinasc2009}, à data do documento, não existiam rotinas previstas no sistema para a avaliação da acurácia das informações. No entanto, autores brasileiros têm se debruçado sobre o tema, comparando principalmente as informações inseridas no Sinasc com dados do prontuário médico ou por meio de entrevistas com as mães. \citeonline{almeida2006validade}, por exemplo, mostraram problemas de acurácia nas informações sobre as condições socioeconômicas (grau de instrução/escolaridade, estado civil) das mães.

Outro aspecto fundamental para a qualidade do Sinasc é avaliar se os documentos estão sendo preenchidos de maneira adequada, saber quantos dos registros existentes no sistema apresentam informação
ou, então, saber a proporção de registros com informação em branco ou ignorada, i.e., a completude. Altos níveis de incompletude podem indicar a necessidade de treinamento para preenchimento de determinadas variáveis, que podem estar gerando dúvidas ou mesmo que não estão sendo consideradas relevantes pelos responsáveis pelo preenchimento da DNV.  

\citeonline{completude_sinasc} avaliaram, a partir dos dados tabulados do Sinasc\footnote{Extraídos a partir do tabulador de dados do Sinasc: \href{https://datasus.saude.gov.br/informacoes-de-saude-tabnet/}{Tabnet.}} a completude de algumas variáveis ao longo do tempo. Demonstraram que, de maneira geral, as variáveis disponibilizadas no tabulador demonstraram uma tendência de queda para os campos marcados como “ignorado”. Avaliando a completude por meio de uma métrica composta das médias das proporções de grupos de variáveis, onde se calculou a média das médias dos grupos para fazer uma comparação entre as Unidades Federativas (UFs). Identificaram que, nos anos em que algumas variáveis foram implementadas, como Consultas pré-natal e raça/cor, houve picos de não preenchimento. A região Norte e Nordeste apresentaram maiores proporções de campos ignorados no início do período analisado (1994), porém melhorando a cobertura a medida do tempo. 

\begin{grafico}
    \centering
     \caption{Métrica de qualidade do preenchimento dos dados de nascimentos por UF e ano}
    \includegraphics[width=6in]{imagens/o-que-se-tem.PNG}
     \legend{\footnotesize Legenda: Valor zero representa preenchimento de todos os campos. Quanto mais distante de zero, maior o uso das categorias “ignorado” e afins.}
    \fonte{\citeonline{completude_sinasc}}
\end{grafico}


Apesar de \citeonline{completude_sinasc} identificarem uma melhora no preenchimento das variáveis do Sinasc ao longo do tempo, sua análise não contempla a variável para idade do pai ou responsável, que não é disponibilizada através do tabulador Tabnet, apenas disponível através dos microdados. Embora a completude do sistema, de maneira geral, apresente melhorias, parece haver diferenças regionais e estaduais quanto à observação do campo idade do pai ou responsável, como veremos na próxima seção. Uma hipótese é de que possíveis diferenças na postura das Secretarias Municipais de Saúde (SMS) em relação ao preenchimento da variável podem ter afetado a relevância dada ao quesito.

A versão da DNV em uso foi atualizada em 2021 e é composta por 52 variáveis, distribuídas em oito blocos: I - IDENTIFICAÇÃO DO RECÉM-NASCIDO (seis variáveis: data, hora, sexo, raça/cor do recém-nascido, peso ao nascer, Índice de Apgar, detectada alguma anomalia congênita), II - LOCAL DA OCORRÊNCIA (sete variáveis: local da ocorrência, estabelecimento, endereço da ocorrência, CEP, bairro/distrito, município de ocorrência, UF), III - PARTURIENTE (14 variáveis: nome, cartão SUS, escolaridade (última série concluída), ocupação habitual, data nascimento, idade, naturalidade, situação conjugal, raça/cor, residência, logradouro, CEP, bairro/distrito, município, UF), IV -RESPONSÁVEL LEGAL (duas variáveis: nome, idade), V - GESTAÇÃO E PARTO (11 variáveis: histórico gestacional, data da última menstruação (DUM), número de semanas de gestação, número de consultas de pré-natal, 34 Mês de gestação em que iniciou o pré-natal, tipo de gravidez, apresentação\footenote{Com relação ao recém nascido. As opções são: 1 – Cefálica; 2 – Pélvica ou Podálica; 3 – Transversa 9 – Ignorado.}, o trabalho de parto foi induzido?, tipo de parto, cesárea ocorreu antes do trabalho de parto iniciar, nascimento assistido por), VI - ANOMALIA CONGÊNITA (uma variável de campo aberto para descrever todas as anomalias congênitas observadas no recém-nascido. ), VII - PREENCHIMENTO (seis variáveis: data do preenchimento, nome do responsável pelo preenchimento, função, tipo documento, número do documento, órgão emissor), VIII - CARTÓRIO (cinco variáveis de preenchimento exclusivo do Oficial do Registro Civil (cartórios): cartório, registro, data, município, UF)\cite{BRmanualDNV2022}.

A disponibilização do banco de dados consolidado ocorre a cada dois anos, devido aos processos de aprimoramento e qualificação dos dados de natalidade. Esses processos são realizados em três etapas junto aos estados e municípios onde há procedimentos de crítica dos dados, mediante checagem de inconsistências e possíveis duplicidades\footnote{\href{https://svs.aids.gov.br/daent/centrais-de-conteudos/dados-abertos/sinasc/}{Sistema de Informações sobre Nascidos Vivos (SINASC) - Nota Informativa.}}. Por esse motivo, nossa análise se limitará ao ano de 2022.
